% Template for Cogsci submission with R Markdown

% Stuff changed from original Markdown PLOS Template
\documentclass[10pt, letterpaper]{article}

\usepackage{cogsci}
\usepackage{pslatex}
\usepackage{float}
\usepackage{caption}

% amsmath package, useful for mathematical formulas
\usepackage{amsmath}

% amssymb package, useful for mathematical symbols
\usepackage{amssymb}

% hyperref package, useful for hyperlinks
\usepackage{hyperref}

% graphicx package, useful for including eps and pdf graphics
% include graphics with the command \includegraphics
\usepackage{graphicx}

% Sweave(-like)
\usepackage{fancyvrb}
\DefineVerbatimEnvironment{Sinput}{Verbatim}{fontshape=sl}
\DefineVerbatimEnvironment{Soutput}{Verbatim}{}
\DefineVerbatimEnvironment{Scode}{Verbatim}{fontshape=sl}
\newenvironment{Schunk}{}{}
\DefineVerbatimEnvironment{Code}{Verbatim}{}
\DefineVerbatimEnvironment{CodeInput}{Verbatim}{fontshape=sl}
\DefineVerbatimEnvironment{CodeOutput}{Verbatim}{}
\newenvironment{CodeChunk}{}{}

% cite package, to clean up citations in the main text. Do not remove.
\usepackage{apacite}

% KM added 1/4/18 to allow control of blind submission
\cogscifinalcopy

\usepackage{color}

% Use doublespacing - comment out for single spacing
%\usepackage{setspace}
%\doublespacing


% % Text layout
% \topmargin 0.0cm
% \oddsidemargin 0.5cm
% \evensidemargin 0.5cm
% \textwidth 16cm
% \textheight 21cm

\title{English Negative Constructions and Communicative Functions in Child
Language}


\author{{\large \bf Zoey Liu (ying.liu.5@bc.edu)} \\ Department of Computer Science \\ Boston College \AND {\large \bf Masoud Jasbi (jasbi@ucdavis.edu)} \\ Department of Linguistics \\ University of California, Davis}


\begin{document}

\maketitle

\begin{abstract}
How does abstract linguistic negation develop in early child language?
Previous research has suggested that abstract negation develops in
stages and from more concrete communicative functions such as rejection,
prohibition, or non-existence. The evidence for the emergence of these
functions in stages is mixed, however, leaving the possibility that
negation is an abstract concept from the beginning that can serve
multiple specific functions depending on early communicative
environment. Leveraging automatic annotations of large-scale child
speech corpora in English, we examine the production trajectores of
seven negative constructions that tend to convey communicative functions
previously discussed in the literature. The results demonstrate the
emergence and gradual increase of these constructions in child speech
within the age range of 18-36 months. Production mostly remains stable,
regular, and close to parents' levels after this age range. These
findings are consistent with two hypotheses: first, that negation starts
as an abstract concept that can serve multiple functions from the
beginning; and second, that negation develops in stages from specific
communicative functions but this development is early and quick, leaving
our corpus methods incapable of detecting them from the available corpus
data.

\textbf{Keywords:}
negation; syntactic construction; communicative function; development;
child language.
\end{abstract}

\hypertarget{introduction}{%
\section{Introduction}\label{introduction}}

Negation is an abstract concept that serves different communicative
functions in everyday communication. A coffee shop can divide its menu
into ``coffee'' and ``not coffee'' sections, with ``not coffee''
bringing together diverse items with no common label like tea and hot
chocolate. It could be used in a sign like ``no mask, no entry'' to
regulate people's behaviors. An employee could say ``I don't like
Mondays'' to communicate their desires or dislikes. But how does
abstract multi-functional negation emerge and develop in the human mind?
Are early stages of negation in child language specific to one or a few
functions? Or does negation emerge as an abstract and multifunctional
concept from the beginning?

Previous literature has proposed that abstract negation develops from
less abstract communicative functions in ordered stages (Pea, 1978). For
instance, Darwin (1872) hypothesized that the earliest manifestation of
negation in infants is when they refuse or reject food from parents by
withdrawing their heads laterally. Similarly, Pea (1978) also proposed
``rejection'' as the first function of negation in child language. By
contrast, Bloom (1970) argued that the use of negation to express
``non-existence'' emerges before ``rejection''. For example, when an
object that children expect to be present is not, children may say:
``there is no window''. Follow-up study by Choi (1988) argued that
``prohibition'' emerges as early as rejections and non-existence. In
cases of prohibition, children use negation to stop others or themselves
from performing actions (e.g.~``don't go''). A function similar to
prohibition is ``inability'' (e.g.~``I cannot zip it''), in that both
involve conceptualizing actions and negating them. Choi (1988) suggested
that expressions of inability emerge after the functions in the first
phase, namely non-existence, rejection, and prohibition.

Despite considerable research on early communicative functions of
negation, their developmental trajectories in children's production have
remained unclear. Recently, Nordmeyer \& Frank (2018) looked at the
speech of five children in the Providence corpus (Demuth, Culbertson, \&
Alter, 2006) and found a great deal of individual variation in how early
a negative function is attested. They reported that the developmental
trajectory of negation in their study was not as consistent as
previously claimed. This leaves the possibility that negation develops
as an abstract concept that can serve multiple communicative functions
early in the development based on the context of use in parent-child
interactions. Therefore, across (a larger number of) children, distinct
functions of negation could develop within the same age range and share
common production trajectories.

However, previous experiments have mainly relied on manual annotations
of corpus data to determine the communicative function of a given
negative utterance, which in turn has limited their work to only a
handful of children per study. Here we aim to go beyond existing work
via utilizing a large collection of child speech corpora in English
(MacWhinney, 2000) along with computational tools to automatically
identify negative utterances that tend to convey the communicative
functions discussed in prior research (Table 1). In particular, our
study investigates three questions: (1) how does the developmental
trajectory of the negative constructions for each function look like?
(2) for utterances expressing the same function, does the developmental
trajectory differ depending on particular lexical items that negation
modifies (e.g.~\emph{like} or \emph{want} for rejection)? (3) taking all
functions into account, do they share similar developmental
characteristics, or would there be function-specific differences?

Given the automatic fashion of our approach, we focus on larger/longer
negative constructions at the single-sentence level. This is in
opposition to short negative forms at the discourse-level such as cases
consisting of one morpheme (e.g.~``no!'') or repetition of negative
morphemes (e.g.~``no no no''), which arguably could express multiple
functions when not taking the discourse context into account and
accordingly leave more room for ambiguous interpretation. Therefore the
negative utterances in our study do not fully cover all negation
instances from the corpora investigated, nor reflect all possible
communicative functions that could be played by negation more broadly,
but it could provide at least a conservative estimate of the age range
during which negation is developed gradually in child production.

\begin{table*}[h]
\small
\centering
\begin{tabular}{rrr}
  \hline
 \textbf{Function} & \textbf{Linguistic Composition} & \textbf{Examples} \\
  \hline
Rejection & with \textit{like} or \textit{want} & \textit{I not like it}, \textit{not want it}  \\
Non-existence & expletives & \textit{there is no soup} \\
Prohibition & with imperative subjectless \textit{do} & \textit{do not spill milk} \\
Inability & with modal \textit{can} & \textit{I cannot zip it} \\
Labeling & modifying nominal or adjectival predicatives & \textit{that's not a crocodile}; \textit{it's no interesting} \\
Epistemic negation & with \textit{know}, \textit{think}, \textit{remember}  & \textit{I not know} \\
Possession & with \textit{have}; or possesive pronouns & \textit{not have the toy}; \textit{not mine} \\
   \hline
\end{tabular}
\caption{Communicative functions of negation in early child language of English.}
\end{table*}

\hypertarget{experiments}{%
\section{Experiments}\label{experiments}}

\hypertarget{data-and-preprocessing}{%
\subsection{Data and preprocessing}\label{data-and-preprocessing}}

For developmental production data of child speech in English, we turned
to the CHILDES database (MacWhinney,
2000)\footnote{Code and data are in quarantine at \url{https://github.com/zoeyliu18/Negative_Constructions.}}
and focused on children with typical development within the age range of
12 - 72 months. Parents' and children's utterances were extracted via
the childes-db (Sanchez et al., 2019) interface using the programming
language R. Structures containing the morphemes \emph{no}, \emph{not}
and \emph{n't} were identified as negative. Cases with a single
\emph{no} or such repetitions (e.g.~\emph{no no no}!) were excluded. In
order to obtain (morpho)syntactic representations for parents' and
children's utterances, we used the dependency grammar framework
(Tesnière, 1959). Dependency relations for all negative utterances were
automatically derived with DiaParser (Attardi, Sartiano, \& Yu, n.d.), a
dependency parsing system that has been demonstrated to achieve
excellent performance for English. To facilitate identification of
negative constructions, we also utilized the available part-of-speech
(POS) information initially provided by CHILDES (Sagae, Davis, Lavie,
MacWhinney, \& Wintner, 2010) when necessary.

In this study, we consider seven communicative functions of negation
shown in Table 1. For each function, using our parsed data set, we
characterized the syntactic features of the negative construction
associated with it. Based on these features, negative utterances were
automatically extracted in a rule-based fashion with the help of POS
information and syntactic dependencies.

\hypertarget{measures}{%
\subsection{Measures}\label{measures}}

As indexes of the developmental trajectory for negative constructions
and their communicative functions in child speech, we measured the
following two metrics at each given age of the children. The first one
is the \emph{ratio} of negative utterances. For instance, the number of
utterances produced by children at the age of 30 months (not just all
negative constructions at this age) is 52,491 in total. Among these
utterances, negative structures that have the function of inability
occur for 141 times; the ratio for this communicative function at 30
months is then calculated as 141 / 52,491 = 0.003.

Given the noisy nature of child production data in general, and the
facts that there are different numbers of utterances and children at
each age, another measure that we utilized is \emph{moving ratio},
borrowed from the model of moving average in analyses of time series
data (Wei, 2006). For a communicative function, the goal of the moving
ratio is still to reflect the production of the negative utterances at
the given age; meanwhile it takes into account the previous production
of all negative constructions of the same function before the specified
age. This would allow us to have a more balanced look at individual
developmental stage (e.g.~age) of a communicative function, in relation
to its development patterns thus far.

The computation of the moving ratio is as follows. For instance, given
that the number of negative utterances that express inability in child
speech is 141 at the age of 30 months, we: (1) count the total number of
negative constructions with the same function produced by children
\emph{at and before} 30 months old (682); (2) compute the total number
of utterances (419,949) within the same age range; (3) divide the number
of (1) by that of (2) (682 / 419,949 = 0.002).

While our focus is negative utterances in child production, we used
parents' speech as comparative references. Therefore for every
communicative function, the same two ratio measures were calculated for
parent speech in a similar fashion. Our plots accordingly contrast the
ratio / moving ratio of different negative constructions between
children's and parents' production at corresponding ages of the
children.

In what follows, we describe in detail the results of each communicative
function and their negative constructions. While we computed both ratio
and moving ratio for every function, our analyses mainly rely on the
latter.

\hypertarget{communicative-functions-of-negative-constructions}{%
\subsection{Communicative functions of negative
constructions}\label{communicative-functions-of-negative-constructions}}

\hypertarget{rejection}{%
\subsubsection{Rejection}\label{rejection}}

For the function of rejection, we examined cases where the lemma of the
head verb of the phrase is either \emph{like} or \emph{want}, and the
head verb is modified by one of the three negative morphemes. Other than
expressions that the speakers used to describe their own desires with
(e.g.~(1)) or without (e.g.~(2)) an auxiliary verb, we also included
cases that express rhetorical inquiries of desires from one interlocutor
addressed to another (e.g.~(3)), and instances where the speaker is
describing the desires of somebody else (e.g.~(4)). This resulted in a
total of 17,436 negative utterances (child: 7,395; parent: 10,041).

~ (1) \emph{I no like sea}

~ (2) \emph{don't wanna go}

~ (3) \emph{don't you wanna try it}

~ (4) \emph{Sarah doesn't like that either}

\begin{figure*}[h]

\begin{CodeChunk}


\begin{center}\includegraphics{figs/emotion-1} \end{center}

\end{CodeChunk}
\caption[This image spans both columns]{Rejection.}\label{fig:rejection}
\end{figure*}

As presented in Figure 1, the overall pattern for children's usage of
negative morphemes for rejection is comparable regardless of the
particular head verb. Comparing child and parent speech, it seems that
children's production of rejection is gradually increasing between the
age of 18 to 36 months. And the production moving ratio in child speech
appears to be more comparable to that of parent speech after 32-34
months.

\hypertarget{non-existence}{%
\subsubsection{Non-existence}\label{non-existence}}

For the function of non-existence, we extracted utterances that have
expletives marked by \emph{there} (e.g.~(5) and (6)), and that the
predicate modified by the negative morphemes is a nominal phrase (headed
by either nouns or pronouns). This led to a total of 1,611 negative
utterances (child: 406; parent: 1,205).

~ (5) \emph{there's no (more) water}

~ (6) \emph{there isn't it}

In child speech, the production of negative constructions to express
non-existence is gradually increasing from 25 to 36 months (Figure 2),
which is by contrast later than that for the communicative function of
rejection presented in Figure 1. This observation does not seem to align
with Bloom (1970), which initially proposed that the development of
non-existence is earlier than that of rejection. On the other hand,
children's production moving ratio gradually approaches that in parent
speech at 36-38 months.

Notice that there appears to be fluctuations of moving ratios between
the age of 19 and 25 months regarding child production. A closer
inspection of the data reveals that within that age range, the frequency
of negative utterances at most ages is either one or zero. Therefore
while the number of total utterances increases along the developmental
trajectory, the moving ratio for negative utterances actually decreases.

\begin{figure*}[h]

\begin{CodeChunk}


\begin{center}\includegraphics{figs/existence-1} \end{center}

\end{CodeChunk}
\caption[This image spans both columns]{Non-existence.}\label{fig:non-existence}
\end{figure*}

\hypertarget{prohibition}{%
\subsubsection{Prohibition}\label{prohibition}}

For constructions that articulate the function of prohibition, we
focused on cases that are annotated as imperatives from the initial
CHILDES annotations. These utterances do not take any subject; the
negative morphemes are combined with the auxiliary verb \emph{do}
(\emph{do}, \emph{does}, \emph{did}) and they together modify the head
verbs of the sentences. In order to not overlap with rejection,
non-existence, epistemic negation and possession (see below), our search
excluded cases where the head verb has any of the following lemma forms:
\emph{like}, \emph{want}, \emph{know}, \emph{think}, \emph{remember},
\emph{have}. This resulted in a total of 938 negative utterances (child:
267; parent: 671).

Based on Figure 3, children are combining negative morphemes for
prohibition more and more regularly between 24-36 months, which is
comparable to that of the function of non-existence, but slightly later
than that of rejection. In comparison, the production moving ratio in
child speech for prohibition is consistently lower than that in parent
speech at any age of the children.

~ (7) \emph{don't blame Charlotte}

\begin{figure*}[h]

\begin{CodeChunk}


\begin{center}\includegraphics{figs/prohibition-1} \end{center}

\end{CodeChunk}
\caption[This image spans both columns]{Prohibition.}\label{fig:prohibition}
\end{figure*}

\begin{figure*}[h]

\begin{CodeChunk}


\begin{center}\includegraphics{figs/inability-1} \end{center}

\end{CodeChunk}
\caption[This image spans both columns]{Inability.}\label{fig:inability}
\end{figure*}

\begin{figure*}[h!]

\begin{CodeChunk}


\begin{center}\includegraphics{figs/learning-1} \end{center}

\end{CodeChunk}
\caption[This image spans both columns]{Labeling.}\label{fig:labeling}
\end{figure*}

\hypertarget{inability}{%
\subsubsection{Inability}\label{inability}}

For the function of inability, we analyzed instances where the negative
morphemes co-occur with the auxiliary \emph{can} (\emph{can} and
\emph{could}; e.g.~(8)) and both of them modify the head verbs of the
utterances. Again, we filtered out cases where the head verbs are the
focus for other functions. Cases without a subject (e.g.~``can't play'')
or where the subject is not \emph{I} (e.g.~``you can't do that'') could
yield ambiguous readings when not looking at a larger discourse context;
they could be a rhetorical question or also express the concept of
prohibition. Therefore to potentially avoid less ambiguity, we
restricted our analyses only to cases with a subject \emph{I}. This led
to 6,369 negative utterances (child: 3,237; parent: 3,132).

~ (8) \emph{I can't see}

As shown in Figure 4, the developmental trajectory of inability is
similar to that of rejection. Negation is applied more and more
regularly between 18-36 months. By contrast, it is different from those
of non-existence and prohibition. It seems that the production
trajectories of the latter two are both becoming more regular at a later
age (25 and 24 months, respectively).

\hypertarget{labeling}{%
\subsubsection{Labeling}\label{labeling}}

To capture the function of labeling, we concentrated on cases where
negative morphemes indicate the identity (e.g.~(9)), and/or
characteristics (e.g.~(10)) of a predicative nominal. We also included
instances where negation is used to modify a predicative adjective
(e.g.~(11)). Utterances where negative morphemes modify a nominal or
adjectival predicate of a copula verb were extracted. None of the
utterances contained expletives (e.g.~``there is no book'') to
distinguish from non-existence. This resulted in a total of 32,474
negative utterances (Child: 4,180; Parent: 28,294).

~ (9) \emph{that's not a farmer}

~ (10) \emph{I'm not a heavy baby Mum}

~ (11) \emph{It's no good}

Based on Figure 5, the developmental pattern for labeling is comparable
to non-existence and prohibition; children are increasing their use of
the negative morphemes around the age range of of 22-36 months.

\hypertarget{epistemic-negation}{%
\subsubsection{Epistemic negation}\label{epistemic-negation}}

Previous studies have reported instances where negative morphemes are
combined with mental/epistemic state verbs such as \emph{know},
\emph{think}, and \emph{remember} in child speech to express epistemic
negation. Here we focused on these three verbs and analyzed negative
utterances that articulate the concept of not knowing (e.g.~(12)) or
uncertainty (e.g.~(13)). The verbs in these cases are modified by the
negative morphemes directly or by the combination of negation with
auxiliaries. Instances where the speaker asks about or describes the
negative epistemic state of another speaker (e.g.~(14)) were also
selected, leading to 21,844 negative utterances in total (child: 4,074;
parent: 17,770).

~ (12) \emph{I not know} / \emph{I didn't remember}

~ (13) \emph{I don't think so}

~ (14) \emph{don't you remember} / \emph{She doesn't know this}

\begin{figure*}[h!]

\begin{CodeChunk}


\begin{center}\includegraphics{figs/epistemic-1} \end{center}

\end{CodeChunk}
\caption[This image spans both columns]{Epistemic negation.}\label{fig:epistemic}
\end{figure*}

\begin{figure*}[h!]

\begin{CodeChunk}


\begin{center}\includegraphics{figs/possession-1} \end{center}

\end{CodeChunk}
\caption[This image spans both columns]{Possession.}\label{fig:possession}
\end{figure*}

\begin{figure*}[h!]

\begin{CodeChunk}


\begin{center}\includegraphics{figs/all-1} \end{center}

\end{CodeChunk}

\caption[This image spans both columns]{All functions; ratio measures plotted here are moving ratios.}\label{fig:all}
\end{figure*}

Based on the data analyzed here (Figure 6), the production of negative
utterances headed by \emph{know} are becoming more regular at an earlier
age (17-18 months) compared to that of \emph{remember}
(\textasciitilde19 months) or \emph{think} (\textasciitilde20 months).
Overall the production moving ratio of utterances with \emph{know} is
comparatively the highest.

\hypertarget{possession}{%
\subsubsection{Possession}\label{possession}}

The last function we explored is possession. We selected cases where the
negative morphemes are combined with auxiliary verbs to modify a head
verb with the lemma form \emph{have} (e.g.~(15)). We also included
individual noun phrases with possessive pronouns as heads and modified
by negative morphemes (e.g.~(16)). Cases in which the syntactic head of
the negative morphemes is a predicate of a copula verb (e.g.~``this is
not mine'') were excluded to separate them from the function
``labeling''. The number of negative utterances that were subjected to
analysis for this function is 8,187 (child: 2,331; parent: 5,856).

~ (15) \emph{I don't have it}

~ (16) \emph{not mine}

Given Figure 7, the developmental trajectory for possession in child
speech appears to have notable differences depending on what the
negative morphemes are modifying. When their syntactic head is
\emph{have}, the pattern is comparable to those of ``rejection'' and
``labeling'', where children are increasing their combination of
negative morphemes from 18 to 36 months. However, the production moving
ratio for utterances headed by possessive pronouns seems to be
relatively stable across different ages.

\hypertarget{discussion}{%
\section{Discussion}\label{discussion}}

Using automatic annotations of large-scale corpora of child-parent
interactions, we presented production trajectories for seven negative
constructions that tend to express rejection, non-existence,
prohibition, inability, labeling, epistemic states, and possession
(Table 1). The results suggest that the production of almost all these
negative constructions (except for prohibition) emerges and gradually
increases within the 18-36 months age range (Figure 8). Their production
frequencies remain stable and regular after 36 months and relatively
close to parents' levels of production. It is important to note that
similar to prior studies, our conclusions are limited to negation in
children's production. Systematic experiments testing children's
comprehension of negative utterances with different communicative
functions are necessary to better understand the origins and
developmental trajectory of negation.

For future work, we would like to explore several directions. First, to
more thoroughly examine and potentially model the developmental
trajectories of negation in child production, certain
production-specific factors (e.g.~length of utterance, ease of
pronunciation) should be taken into account as well. In addition, we aim
to investigate the production trajectory of positive counterparts to our
negative structures (e.g.~``I know'' for ``I don't know''). Comparisons
of negative utterances in relation to their positive counterparts would
allow us to further analyze the developmental paths of negation within
specific constructions.

Lastly, our experiments have concentrated on larger syntactic structures
at the utterance level, hence cases where negation is used as discourse
markers to respond to previous utterance(s) were excluded. However,
these instances also have important semantic and conceptual roles in the
communication between children and parents (e.g.~parent: \emph{do you
want some bread?}; child: \emph{no no no}). Thus inclusions of negative
structures at a more comprehensive level would be able to paint a more
clear picture about the development of negation.

\hypertarget{references}{%
\section{References}\label{references}}

\setlength{\parindent}{-0.1in} 
\setlength{\leftskip}{0.125in}

\noindent

\hypertarget{refs}{}
\leavevmode\hypertarget{ref-diaparser}{}%
Attardi, G., Sartiano, D., \& Yu, Z. (n.d.). DiaParser attentive
dependency parser. \emph{Submitted for Publication}.

\leavevmode\hypertarget{ref-bloom1970language}{}%
Bloom, L. M. (1970). \emph{Language development: Form and function in
emerging grammars} (PhD thesis). Columbia University.

\leavevmode\hypertarget{ref-choi1988semantic}{}%
Choi, S. (1988). The semantic development of negation: A
cross-linguistic longitudinal study. \emph{Journal of Child Language},
\emph{15}(3), 517--531.

\leavevmode\hypertarget{ref-darwin1872expression}{}%
Darwin, C. (1872). \emph{The expression of the emotions in man and
animals}. John Murray.

\leavevmode\hypertarget{ref-demuth2006word}{}%
Demuth, K., Culbertson, J., \& Alter, J. (2006). Word-minimality,
epenthesis and coda licensing in the early acquisition of English.
\emph{Language and Speech}, \emph{49}(2), 137--173.

\leavevmode\hypertarget{ref-macwhinney2000childes}{}%
MacWhinney, B. (2000). \emph{The childes project: Tools for analyzing
talk. Transcription format and programs} (Vol. 1). Psychology Press.

\leavevmode\hypertarget{ref-nordmeyer2018individual}{}%
Nordmeyer, A., \& Frank, M. C. (2018). Individual variation in
children's early production of negation. In \emph{Proceedings of the
40th annual meeting of the cognitive science society} (pp. 2167--2172).

\leavevmode\hypertarget{ref-pea1978}{}%
Pea, R. (1978). \emph{The development of negation in early child
language} (PhD thesis). University of Oxford.

\leavevmode\hypertarget{ref-sagae2010morphosyntactic}{}%
Sagae, K., Davis, E., Lavie, A., MacWhinney, B., \& Wintner, S. (2010).
Morphosyntactic annotation of childes transcripts. \emph{Journal of
Child Language}, \emph{37}(3), 705--729.

\leavevmode\hypertarget{ref-sanchez2019childes}{}%
Sanchez, A., Meylan, S. C., Braginsky, M., MacDonald, K. E., Yurovsky,
D., \& Frank, M. C. (2019). Childes-db: A flexible and reproducible
interface to the child language data exchange system. \emph{Behavior
Research Methods}, \emph{51}(4), 1928--1941.

\leavevmode\hypertarget{ref-dg}{}%
Tesnière, L. (1959). \emph{Éléments de syntaxe structurale}. Paris:
Klincksieck.

\leavevmode\hypertarget{ref-wei2006time}{}%
Wei, W. W. (2006). Time series analysis. In \emph{The oxford handbook of
quantitative methods in psychology: Vol. 2}.

\bibliographystyle{apacite}


\end{document}
